\documentclass[xcolor=svgnames]{beamer}
\usepackage[british]{babel}
\usepackage[T1]{fontenc}
\usepackage[utf8]{inputenc}

\usepackage{tcolorbox}
\usepackage{graphicx}
\usepackage{booktabs}

%\usetheme[block=fill,progressbar=frametitle]{metropolis}
\usepackage{lmodern}
\usepackage{textgreek}
\usepackage{bytefield}
\usepackage{tikz}
\usetikzlibrary{shapes.misc, shapes.symbols, positioning, calc}
\usetikzlibrary{arrows.meta}

\usepackage{xcolor}

\usepackage{siunitx}
\usepackage{tikz-timing}
\usepackage[european]{circuitikz}

\useinnertheme{default}
\useoutertheme{infolines}
\usecolortheme{seahorse}
\setbeamercolor*{alerted text}{fg=blue!75!black}
\setbeamertemplate{blocks}[rounded]
\setbeamertemplate{itemize item}[triangle]
\setbeamertemplate{itemize subitem}[circle]
\setbeamertemplate{itemize subsubitem}[square]

\usecolortheme{rose}
\definecolor{NUblue}{RGB}{62,141,165}
\definecolor{NUbluedark}{RGB}{40,119,143}
\setbeamercolor*{palette primary}{use=structure,fg=white,bg=NUblue}
\setbeamercolor*{palette quaternary}{fg=white,bg=NUbluedark}
\setbeamercolor{section in head/foot}{fg=white,bg=NUbluedark}
\setbeamercolor{subsection in head/foot}{fg=white,bg=NUblue}
\setbeamercolor{frametitle}{fg=NUbluedark!150,bg=NUblue!40}
\setbeamercolor{title in head/foot}{fg=white,bg=NUblue}
\setbeamercolor{author in head/foot}{fg=white, bg=NUbluedark}
\setbeamercolor{date in head/foot}{fg=white, bg=NUblue!60}
\setbeamercolor{title}{fg=NUbluedark!150,bg=NUblue!30}
\setbeamercolor{date}{fg=NUbluedark!150}
\setbeamercolor{block title}{fg=white,bg=NUblue}

\title{Control}
\subtitle{Control systems and Computer Networks}

\author{Dr Alun Moon}
\date{Lecture 8.2}

\tikzset{onslide/.code args={<#1>#2}{%
\only<#1>{\pgfkeysalso{#2}} % \pgfkeysalso doesn't change the path
}}
\tikzset{temporal/.code args={<#1>#2#3#4}{%
\temporal<#1>{\pgfkeysalso{#2}}{\pgfkeysalso{#3}}{\pgfkeysalso{#4}} % \pgfkeysalso doesn't change the path
}}


\begin{document}
\frame{\maketitle}

\begin{frame}{Digital Control}
    \begin{itemize}
        \item The majority of embedded systems provide some form of control
        \item Monitor some property of a system
        \item Decide on what value it \emph{should} be
        \item Drive the inputs to the system in such a way, so that the property reaches it's desired value.
    \end{itemize}
\end{frame}

\begin{frame}{Some terms}
\begin{description}
    \item[Plant] The system you want to control (older term)
    \item[System] the thing controlled
    \item[Process variable] The property measured,
    \item[Set-point] the desired value of the \emph{process-variable}
    \item[Control input] some property that the controller can affect, causeing a change in the monitored \emph{process variable}
    \item[Controller] The algorithm and hardware that decides what inputs to use to move the \emph{process variable} to the \emph{set point}
\end{description}
\end{frame}

\setlength{\unitlength}{2cm}
\begin{frame}{Basic view of a system}
    \begin{picture}(3.25,0.5)(0,-0.25)
    \thicklines
    % dpic version 2018.02.01 option (none, LaTeX picture assumed) for LaTeX
    \put(0,-0.25){\makebox(0.75,0.5){\rlap{\hbox to 2.5bp{}input}}}
    \put(1.15,0){\vector(1,0){0.1}}
    \put(0.75,0){\line(1,0){0.4}}
    \put(1.25,-0.25){\framebox(0.75,0.5){system}}
    \put(2.4,0){\vector(1,0){0.1}}
    \put(2,0){\line(1,0){0.4}}
    \put(2.5,-0.25){\makebox(0.75,0.5){response}}
    \end{picture}

\begin{itemize}
    \item A system can be anything, from the heating of a house, speed of a car,\ldots
    \item Some \emph{inputs} can be changed
    \begin{itemize}
        \item The heating can be turned on or off
        \item The throttle can be opened or closed
    \end{itemize}
    \item The system responds in some way that can be measured, temperature, speed, pressure, \ldots
\end{itemize}
\end{frame}

\begin{frame}{Control loop}
    \begin{picture}(5.04,0.752)(0,-0.502)
    \thicklines
    % dpic version 2018.02.01 option (none, LaTeX picture assumed) for LaTeX
    \put(0,-0.25){\makebox(0.75,0.5){setting}}
    \put(1.15,0){\vector(1,0){0.1}}
    \put(0.75,0){\line(1,0){0.4}}
    \put(1.25,-0.25){\framebox(0.75,0.5){control}}
    \put(2.4,0){\vector(1,0){0.1}}
    \put(2,0){\line(1,0){0.4}}
    \put(2.5,-0.25){\framebox(0.75,0.5){system}}
    \put(3.25,0){\line(1,0){0.5}}
    \put(3.77,0){\circle{0.04}}
    \put(4.19,0){\vector(1,0){0.1}}
    \put(3.79,0){\line(1,0){0.4}}
    \put(4.29,-0.25){\makebox(0.75,0.5){\shortstack{desired\\%
    temperature}}}
    \put(3.77,0){\line(0,-1){0.5}}
    \put(3.77,-0.501){\circle{0.002}}
    \put(3.77,-0.502){\line(-1,0){2.145}}
    \put(1.625,-0.35){\vector(0,1){0.1}}
    \multiput(1.625,-0.501)(0,0.005033){31}{\makebox(0.005555,0.00825){\tiny .}}
    \end{picture}

Adding a control loop
\begin{itemize}
    \item Monitor the temperature
    \item compare with the desired temperature
    \item do something to the system to change the temperature as needed
\end{itemize}
\textbf{Feedback control}
\end{frame}

\begin{frame}
\begin{block}{We're not going too deep}
    \begin{itemize}
        \item You can do whole Master's degrees on Control Engineering
        \item Complex Mathematics
        \begin{itemize}
            \item Fourier analysis
            \item Laplace transforms
        \end{itemize}
        \item You'll be please to know we're not going there here.
        \item But I want to give you an insight into the principles
    \end{itemize}
\end{block}
\end{frame}


\begin{frame}{feedback controller}
    \begin{picture}(5.74,0.902)(0,-0.502)
    \thicklines
    % dpic version 2018.02.01 option (none, LaTeX picture assumed) for LaTeX
    \put(0,-0.25){\makebox(0.75,0.5){desired}}
    \put(1.15,0){\vector(1,0){0.1}}
    \put(0.75,0){\line(1,0){0.4}}
    \put(1.35,0){\circle{0.2}}
    \put(1.85,0){\vector(1,0){0.1}}
    \put(1.45,0){\line(1,0){0.4}}
    \put(1.95,-0.25){\framebox(0.75,0.5){control}}
    \put(3.1,0){\vector(1,0){0.1}}
    \put(2.7,0){\line(1,0){0.4}}
    \put(3.2,-0.25){\framebox(0.75,0.5){system}}
    \put(3.95,0){\line(1,0){0.5}}
    \put(4.47,0){\circle{0.04}}
    \put(4.89,0){\vector(1,0){0.1}}
    \put(4.49,0){\line(1,0){0.4}}
    \put(4.99,-0.25){\makebox(0.75,0.5){\shortstack{desired\\%
    temperature}}}
    \put(4.47,0){\line(0,-1){0.5}}
    \put(4.47,-0.501){\circle{0.002}}
    \put(4.47,-0.502){\line(-1,0){3.12}}
    \put(1.35,-0.2){\vector(0,1){0.1}}
    \put(1.35,-0.501){\line(0,1){0.301}}
    \put(0.975,-0.1){\makebox(0.75,0.5){error}}
    \end{picture}
\begin{itemize}
    \item Measure property
    \item Check desired value
    \item Calculate error between desired and current value
    \item Use error to decide on what to do, how to change inputs
\end{itemize}

\end{frame}

\begin{frame}{An Example}{Domestic heating }
\begin{enumerate}
    \item Set temperature on control
    \item Measure room temperature
    \item control:
    \begin{itemize}
        \item if too hot turn heating Off
        \item if too cold turn heating On
    \end{itemize}
    \item Repeat every second, minute or so.
\end{enumerate}
\end{frame}

\begin{frame}{}
\end{frame}

speed of system
bang-bang
proportional
PID
tuning (yuk)


\end{document}
